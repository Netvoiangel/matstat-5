\documentclass{article}
\usepackage{amsmath}
\usepackage{graphicx}
\usepackage[utf8]{inputenc}
\usepackage[russian]{babel}
\usepackage{geometry}
\geometry{a4paper, margin=1in}

\begin{document}

\section{Критерий знаков}

\subsection{Задание 1.1}

Даны две выборки:
\begin{itemize}
    \item $X = \{x_1, \dots, x_n\} \sim N(2, 5)$, где $n = 500$
    \item $Y = \{y_1, \dots, y_n\} \sim N(2, 8)$
\end{itemize}

Необходимо подсчитать знаки:
\begin{itemize}
    \item Если $x_j < y_j$, ставим знак ``-''
    \item Если $y_j < x_j$, ставим знак ``+''
\end{itemize}

Выбираем статистику:
\[ w = \min(\text{число знаков } -, \text{ число знаков } +) \]

Нулевая гипотеза $H_0$: $X - Y = 0$ в среднем (так как операции не определены).

Статистика критерия:
\[ \hat{t} = \frac{w - \frac{n}{2}}{\sqrt{\frac{n}{4}}} \approx N(0, 1) \]

\subsection{Задание 1.2}

Аналогично, но:
\begin{itemize}
    \item $X = \{x_1, \dots, x_n\} \sim N(2, 5)$
    \item $Y$ сдвинута: $Y \sim N(2.2, 7)$
\end{itemize}

Заметим, что сдвиг среднего меньше, чем в п.1.1. Критерию знаков не важна дисперсия, но необходим одинаковый объём выборок.

Аналогично вычисляем статистику, но критерий знаков здесь менее мощный (в этом его отличие). Вероятно, в этой ситуации критерий знаков не сможет обнаружить различие.

\section{Ранговый анализ}

Рассмотрим разности $|x_i - y_i|$ и построим ранговую статистику:

\begin{enumerate}
    \item Упорядочим все разности по возрастанию
    \item Выделим 4 ранговые группы с границами:
    \begin{itemize}
        \item Первый элемент: 1
        \item Границы групп: $n/4$, $n/2$, $3n/4$, $n$
    \end{itemize}
    \item Присвоим ранговые расстояния: 1, 2, 3, 4
\end{enumerate}

\subsection{Критерий Голдфелда-Квандта}

Определим ранги:
\[ R_i = \text{rank}(x_i - y_i) \]

Ранговая статистика:
\[ \hat{t} = \sum_{i=1}^n \text{sgn}(x_i - y_i) \cdot R_i \]

Нормированная статистика:
\[ \frac{\hat{t}}{\sqrt{\frac{n(n+1)(2n+1)}{6}}} \approx N(0,1) \]

где $\text{sgn}(x_i - y_i)$ - знак разности:
\[ \text{sgn}(z) = \begin{cases} 
1 & \text{если } z > 0 \\
0 & \text{если } z = 0 \\
-1 & \text{если } z < 0 
\end{cases} \]
\section{Ранговый коэффициент корреляции Кендалла}

\subsection{Задание 2.1}

Дана выборка:
\[ \{x_1, \dots, x_n\} \sim N(2, 5) \]

\subsubsection{Случай 2.1(а)}
Выборка делится пополам на две части равного размера.

\subsubsection{Служание 2.1(б)}
Задана линейная зависимость с шумом:
\[ y_i = 3x_i - 5 + \varepsilon_i, \quad \varepsilon_i \sim N(0, 1) \]
Получаем пары значений $(x_i, y_i)$.

\subsection{Определение согласованности}

Для двух пар $(x_i, y_i)$ и $(x_j, y_j)$:
\begin{itemize}
    \item Пары \textbf{согласованы}, если:
    \[ (x_i < x_j \text{ и } y_i < y_j) \text{ ИЛИ } (x_j < x_i \text{ и } y_j < y_i) \]
    \item Пары \textbf{не согласованы} в противном случае
\end{itemize}

\subsection{Коэффициент корреляции Кендалла}

Ранговый коэффициент Кендалла вычисляется как:
\[ \tau = \frac{\text{число согласованных пар} - \text{число несогласованных пар}}{\text{общее число пар}} \]

Формальная запись:
\[ \tau = \frac{2}{n(n-1)} \sum_{1 \leq i < j \leq n} \text{sgn}(x_i - x_j) \cdot \text{sgn}(y_i - y_j) \]

где функция знака определена как:
\[ \text{sgn}(z) = \begin{cases} 
1 & \text{если } z > 0 \\
0 & \text{если } z = 0 \\
-1 & \text{если } z < 0 
\end{cases} \]

\subsection{Свойства коэффициента}
\begin{itemize}
    \item $\tau \in [-1, 1]$
    \item $\tau = 1$: полная положительная зависимость
    \item $\tau = -1$: полная отрицательная зависимость
    \item $\tau = 0$: отсутствие монотонной зависимости
\end{itemize}

\subsection{Задание 2.2: Случай слабой зависимости}

Даны:
\begin{itemize}
    \item $X \sim N(2, 5)$
    \item Модель с слабой линейной зависимостью:
    \[ y_i = 0.05x_i - 5 + \varepsilon_i, \quad \varepsilon_i \sim N(0, 1)
    \]
\end{itemize}

Особенности случая:
\begin{itemize}
    \item Очень слабый наклон (0.05) делает зависимость почти незаметной
    \item Критерий Кендалла будет иметь низкую мощность в этом случае
    \item Необходим большой объем выборки $n$ для обнаружения такой зависимости
    \item Стандартная ошибка в знаменателе $\hat{t}$ компенсирует влияние $n$
\end{itemize}
\section{Критерий отсутствия автокорреляции ближайших соседей}

Для выборки $\{x_1, \dots, x_n\}$ рассматриваем коэффициент автокорреляции:

\subsection{Коэффициент автокорреляции Пирсона}
Исправленная формула:
\[
\rho_{1,n} = \frac{n\sum_{j=1}^{n-1}x_jx_{j+1} - \left(\sum_{j=1}^n x_j\right)^2 + nx_1x_n}{n\sum_{j=1}^n x_j^2 - \left(\sum_{i=1}^n x_i\right)^2}
\]

\subsection{Моменты статистики}
\begin{itemize}
    \item Математическое ожидание:
    \[ M[\rho_{1,n}] = -\frac{1}{n-1} \]
    
    \item Дисперсия:
    \[ D[\rho_{1,n}] = \frac{n(n-3)}{(n+1)(n-1)^2} \]
\end{itemize}

\subsection{Нормальное приближение}
Статистика для проверки гипотез:
\[
\hat{t} = \frac{\rho_{1,n} + \frac{1}{n-1}}{\sqrt{D[\rho_{1,n}]}} \approx N(0,1)
\]

\subsection{Задание 3.1 ($n=500$)}
Для выборки:
\[ X \sim N(3, 7) \]
вычисляем статистику $\hat{t}$ по приведенным формулам.

\subsection{Задание 3.2 (модифицированная выборка)}
Вводим преобразование:
\[
\overline{x_j} = \begin{cases}
x_i & \text{при } i = j \\
x_j - 5x_{i-1} + 2\varepsilon & \text{при } 1 \leq i < j \leq n
\end{cases}, \quad \varepsilon \sim N(0,1)
\]

Особенности:
\begin{itemize}
    \item Создается автокорреляционная структура
    \item Коэффициент $-5$ перед $x_{i-1}$ создает сильную зависимость
    \item Добавление шума $2\varepsilon$ делает модель более реалистичной
\end{itemize}

\end{document}
